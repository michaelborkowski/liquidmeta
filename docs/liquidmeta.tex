\documentclass[11pt]{article}
\usepackage{amssymb,amsmath,amsthm}
\usepackage[margin=1.25in]{geometry}
\usepackage{graphicx,ctable,booktabs}
\usepackage{textcomp,stmaryrd}
\usepackage{mathpartir} 

\allowdisplaybreaks

\newtheorem{theorem}{Theorem}%[section]
\newtheorem{lemma}[theorem]{Lemma}
\newtheorem{proposition}[theorem]{Proposition}
\newtheorem{corollary}[theorem]{Corollary}

%%%%% Set up the header
\usepackage{fancyhdr}
\usepackage{extramarks} % fixes the buggy section numbering
\usepackage{comment}
\pagestyle{fancy}
\lhead{Liquid Meta}
\chead{}
\rhead{\thepage}
\renewcommand{\headrulewidth}{.3pt}
\setlength\voffset{-0.25in}
\setlength\textheight{648pt}

\newcommand{\eps}{\varepsilon}
\newcommand{\bind}{\hspace{0.05em}{:}\hspace{0.05em}} %x:t w/o space
\newcommand{\col}{\mathbin{:}}       % e : t with a little space
\newcommand{\lb}{\llbracket}         % [[
\newcommand{\rb}{\rrbracket}         % ]]
\newcommand{\step}{\hookrightarrow}
\newcommand{\many}{\hookrightarrow^*}

% purely font faces
\newcommand{\true}{\mathtt{true}}
\newcommand{\Int}{{\sf Int}}
\newcommand{\Bool}{{\sf Bool}}

\newcommand{\existype}[3]{\exists\, #1\bind #2.\, #3}
\newcommand{\functype}[3]{#1\bind #2 \rightarrow #3}
\newcommand{\foralltheta}{\forall\theta.\,\theta\in\lb\Gamma\rb}
\newcommand{\letin}[3]{{\tt let}\,#1\hspace{0.1em}{=}\hspace{0.1em}#2\,{\tt in}\,#3}

%%%%%%%%%%%%%%%%%%%%%%%%%

\begin{document}

\title{Liquid Meta}
\author{\textsc{Michael Borkowski} \\ (summarizing joint work with {\sc Ranjit Jhala} and {\sc Niki Vazou})}
\date{January 9, 2020}

\maketitle
\thispagestyle{empty}

\section{Our language $\lambda_1$}

We work with a simply typed lambda calculus with call-by-value semantics which is augmented by refinement types, dependent function types, and existential types. Our language is based on the language $\lambda$ in Jhala's forthcoming manuscript [Jhala] and incorporates some aspects from the $\lambda^H$ of Vazou et al [VSJ$^+$14]. The existential types were used in a similar setting by Knowles and Flanagan [KF09].

We start with the syntax of term-level expressions in our language:

\begin{align*}
{\sf Values} \;\;\; v :&=\;\: {\tt true}, {\tt false}
                         {\kern 5em}& boolean\; constants\\
                   &\;\;|\quad 0, 1, 2, \ldots 
                         & integer \; constants\\
                   &\;\;|\quad x & variables\\
                   &\;\;|\quad \lambda x . e
                         & abstractions \\
                   &\;\;|\quad e_1 \wedge e_2,\; e_1 \vee e_2,
                         \;\neg e_1 
                         & built{\rm -}in\; primitives \\
                   &\;\;|\quad e_1 \leq e_2,\; e_1 = e_2
                         & built{\rm -}in\; primitives
\end{align*}
\begin{align*}
{\sf Expressions} \;\;\; e :&=\;\: v {\kern 5 em}& values \\
	                &\;\;|\quad e_1\; e_2 & applications \\
	                &\;\;|\quad {\tt let}\; x = e_1
	                      \; {\tt in} \; e_2 & let\; expressions\\
	                &\;\;|\quad e_1 \col t & annotations \\
\end{align*}

Next, we give the syntax of the types and binding environments used in our language:

\begin{align*}
{\sf Base\; types} \quad b :&=\;\;{\sf Bool}{\kern 5em}&booleans\\
                   &\;\;|\quad {\sf Int} &integers \\ \\
{\sf Types} \quad t :&=\;\; b & base \\
                   &\;\;|\quad b\{r\} & refinement \\
                   &\;\;|\quad x\bind t_x \rightarrow t 
                   &dependent\; function\\
                   &\;\;|\quad \exists\, x\bind t_x.\, t 
                   &existential\\ \\
{\sf Refinements} \quad r :&= \{x\col p\}  \\ \\
%\end{align*}
%\begin{align*}
{\sf Environments} \quad \Gamma :&=\;\; \varnothing
                   {\kern 5em}& empty \\
                   &\;\;|\quad \Gamma, x\bind t & bind\;variable\\
\end{align*}

Next, we give the syntax of the Boolean predicates and constraints involved in refinements and subtyping judgments:

\begin{align*}
{\sf Predicates} \;\; p :&=\;\: \{ e \;|\; \exists\, \Gamma.\, 
                   \Gamma \vdash e : {\sf Bool}\}
                   {\kern 3 em}& expressions\; of\; type\; {\sf Bool} \\ \\
{\sf Constraints} \;\; c :&=\;\: p  {\kern 3 em}& predicates\\
                   &\;\;|\quad c_1 \wedge c_2 & conjunction\\
                   &\;\;|\quad\forall\, x\bind b.\, p\Rightarrow c
                   &implication
\end{align*}
Our definition of predicates above departs from the languages of [Jhala] by allowing predicates to be arbitrary expressions from the main language (which are Boolean typed under the appropriate binding environment).
In [Jhala] however, predicates are quantifier-free first-order formulae over a vocabulary of integers and a limited number of relations. We initially took this approach, but were unable to fully define the denotational semantics for this type of language. In particular, when we define closing substitutions we need to define the substitution of a type $\theta(t)$ as the type resulting from $t$ after performing substitutions for all variables bound to expresssions in
$\theta = (x_1 \mapsto e_1, \ldots, x_n \mapsto e_n)$. Substituting arbitrary expressions into $t$ requires substituting arbitrary expressions into predicates, and it isn't clear how to do this for non-values like $((\lambda x. x)\, 3)$ without taking predicates to be all Boolean-typed program expressions. \\

Returning to our $\lambda_1$, we next define the operational semantics of the language. We treat the reduction rules (small step semantics) of the various built-in primitives as external to our language, and we denote by $\delta(c,v)$ a function specifying them. The reductions are defined in a curried manner, so for instance we have that 
$c\; v_1\; v_2 \many \delta(\delta(c,v_1),v_2)$. Currying gives us unary relations like $m\!\!\leq$ which is a partially evaluated version of the $\leq$ relation.
\begin{align*}
\delta(\wedge,\true) &:= \lambda x.\, x &
  \delta(\leq,m) &:= m\!\!\leq \\
\delta(\wedge,{\tt false}) &:= \lambda x.\, {\tt false} &
  \delta(m\!\!\leq, n) &:= {\tt bval}(m \leq n)\\
\delta(\vee,\true) &:= \lambda x.\, \true &
  \delta(=,m) &:= m\!\!= \\
\delta(\vee,{\tt false}) &:= \lambda x.\, x &
  \delta(m\!\!=, n) &:= {\tt bval}(m = n) \\ 
\delta(\neg,\true) &:= {\tt false}\\
\delta(\neg,{\tt false}) &:= \true 
\end{align*}

Now we give the reduction rules for the small-step semantics. In what follows, $e$ and its variants refer to an arbitrary expression, $v$ refers to a value, $x$ to a variable, and $c$  refers to a built-in primitive.
\begin{mathpar}
\inferrule*[Right=E-Prim]{  }{c\; v \step \delta(c,v)} \and
\inferrule*[Right=E-App1]{e \step e'}{e\; e_1 \step e'\; e_1} \\
\inferrule*[Right=E-App2]{e \step e'}{v\; e \step v \; e'} \and
\inferrule*[Right=E-AppAbs]{ }
  {(\lambda x.\, e)\; v \step e[v/x]} \\
\inferrule*[Right=E-Let]{ e_x \step e'_x}
  {\letin{x}{e_x}{e} \step \letin{x}{e'_x}{e}} \and
\inferrule*[Right=E-LetV]{ }{\letin{x}{v}{e} \step e[v/x]} \\
\inferrule*[Right=E-Ann]{e \step e'}{e \col t \step e' \col t}\and
\inferrule*[Right=E-AnnV]{ }{v \col t \step v}
\end{mathpar}

Next, we define the typing rules of our $\lambda_1$.
The type judgments in the language $\lambda_1$ will be denoted $\vdash$ with a colon between term and type. For clarity, we distinguish between this and other judgments by using $\vdash$ with a subscript in most other settings. For instance, the judgement $\Gamma \vdash_w t$ says that type $t$ is well-formed in environment $\Gamma$:
\begin{mathpar}
\inferrule*[Right=WF-Base]{ }{\Gamma \vdash_w b}\and
\inferrule*[Right=WF-Refn]{\Gamma, x\bind b \vdash e : \Bool}
{\Gamma \vdash_w b\{x\col e\}}\\
\inferrule*[Right=WF-Func]
{\Gamma \vdash_w t_x \qquad \Gamma, x\bind t_x \vdash_w t}
{\Gamma \vdash_w x\bind t_x \rightarrow t} \and
\inferrule*[Right=WF-Exis]
{\Gamma \vdash_w t_x \qquad \Gamma, x\bind t_x \vdash_w t}
{\Gamma \vdash_w \exists\, x\bind t_x .\, t} 
\end{mathpar}

The judgement $\vdash_w \Gamma$ says that the environment $\Gamma$ is well formed, meaning that variables are only bound to well-formed types:
\begin{mathpar}
\inferrule*[Right=WFE-Empty]{ }{\vdash_w \varnothing} \and
\inferrule*[Right=WFE-Bind]{\Gamma \vdash_w t_x \quad \vdash_w \Gamma}{\vdash_w \Gamma, x\bind t_x}
\end{mathpar}

Now we give the rules for the typing judgements. As with the reduction rules, we take the type of our built-in primitives to be external to our language. We denote by $ty(c)$ the function that specifies the most specific type possible for $c$. More details on $ty(c)$ are given in the next section.
\begin{mathpar}
\inferrule*[Right=T-Prim]{ty(c) = t}{\Gamma \vdash c : t} \and
\inferrule*[Right=T-Var]{x\bind t \in \Gamma}{\Gamma \vdash x:t}\and
\inferrule*[Right=T-Abs]
{\Gamma, x\bind t_x \vdash e:t \qquad \Gamma\vdash_w t_x}
{\Gamma \vdash \lambda x.\, e \,:\, x\bind t_x \rightarrow t}\\
\inferrule*[Right=T-App]
{\Gamma \vdash e \,:\, x\bind t_x \rightarrow t \qquad \Gamma \vdash e' : t_x}
{\Gamma \vdash e\; e' : \exists\, x\bind t_x.\, t} \and
\inferrule*[Right=T-Let]
{\Gamma \vdash e_x : t_x \quad \Gamma,x\bind t_x \vdash e_2 : t \quad \Gamma \vdash_w t}
{\Gamma \vdash \letin{x}{e_x}{e} : t} \\
\inferrule*[Right=T-Ann]{\Gamma\vdash e:t}{\Gamma\vdash e\col t\,:\,t}
\and \inferrule*[Right=T-Sub]
{\Gamma\vdash e:s \qquad \Gamma\vdash s<:t \qquad\Gamma\vdash_w t}
{\Gamma \vdash e : t}
\end{mathpar}

The last rule, {\sc T-Sub}, uses the subtyping judgement $\Gamma \vdash s <: t$. The subtyping rules are as follows:

\begin{mathpar}
\inferrule*[Right=S-Base]
{\Gamma, x_1\bind b\{x_1\col p_1\} \vdash_e p_2[x_1/x_2]}
{\Gamma \vdash b\{x_1\col p_1\} <: b\{x_2\col p_2\}} \and 
\inferrule*[Right=S-Func]
{\Gamma \vdash s_2 <: s_1 \quad \Gamma, x_2\bind s_2 \vdash t_1[x_2/x_1] <: t_2}
{\Gamma \vdash x_1\bind s_1 \rightarrow t_1 <: x_2\bind s_2\rightarrow t_2} \and
\inferrule*[Right=S-Witn]
{\Gamma \vdash e_x : t_x \qquad \Gamma \vdash t <: t'[e_x/x]}
{\Gamma \vdash t <: \exists\, x\bind t_x.\, t'} \and
\inferrule*[Right=S-Bind]
{\Gamma, x\bind t_x \vdash t <: t' \quad x \not\in free(t')}
{\Gamma \vdash \exists\, x\bind t_x.\, t <: t'}
\end{mathpar}

The first rule above, {\sc S-Base}, uses the entailment judgement $\Gamma \vdash_e p$ which (roughly) states that predicate $p$ is valid (in the sense of a logical formula) when universally quantified over all variables bound in environment $\Gamma$.
We give the inference rule for the entailment judgement:

\begin{mathpar}
\inferrule*[Right=Ent-Pred]
{\forall\, \theta.\, \theta\in \lb\Gamma\rb \Rightarrow \theta(p) \many \true}
{\Gamma \vdash_e p} \and
%\inferrule*[Right=Ent-EmpI]
%{\forall\, \theta.\, \theta \in \lb b\{x\col p\}\rb\Rightarrow \varnothing\vdash_e \theta(c) }
%{\varnothing \vdash_e \forall\, x\bind b.\, p \Rightarrow c} \\
\end{mathpar}

The constraints in the grammar above are only used in the algorithmic typing rules. The constraints represent SMT instances that capture a predicate and its binding environment.

\begin{mathpar}
\inferrule*[Right=Ent-Emp]{p \many \true}
{\textsf{SMT}(p)} \and
\inferrule*[Right=Ent-Conj]
{\textsf{SMT}(c_1) \quad \textsf{SMT}(c_2) }
{\textsf{SMT}(c_1) \wedge {\sf SMT}(c_2)} \\
%\inferrule*[Right=Ent-Emp]{\textsc{Valid}(c)}
%{\varnothing \vdash_e c} \and
\inferrule*[Right=Ent-Witn]
{\textsf{SMT}(\forall\, x\bind b.\, p \Rightarrow c)}
{\textsf{SMT}(p[v/x] \Rightarrow c[v/x])}
\end{mathpar}


\section{Preliminaries}   %%%%%%%%%%% 2222222222222 %%%%%%%%

For clarity, we distinguish between different typing judgments with a subscript.  The type judgments in the underlying typed lambda calculus will be denoted by $\vdash_B$ and a colon before the type. In order to speak about the base type underlying some type, we define a function that erases refinements in types:
\[
\lfloor b\{x:p\} \rfloor := b, \quad
\lfloor x\bind t_x \rightarrow t\rfloor := \lfloor t_x \rfloor \rightarrow \lfloor t \rfloor
, \quad{\rm and}\quad
\lfloor \exists\, x\bind t_x.\, t\rfloor := \lfloor t\rfloor
\]

\begin{comment}
To simplify the meta-theory, we follow [VSJ$^+$14] in extending our language with a non-value expression $\bot$, which does not evaluate but which has any refinement type:
\begin{mathpar}
\inferrule*[Right=T-Bot]{\Gamma \vdash_w b\{x \col p\}}{\Gamma \vdash \bot : b\{x\col p\}}	
\end{mathpar}
We define a predicate {\sf BotLess}$(e)$ which holds if and only if the expression $e$ does not contain $\bot$.
\end{comment}

We start our development of the meta-theory by giving a definition of {\em type denotations}. Roughly speaking, the denotation of a type $t$ is the class of expressions $e$ with the correct underlying base type such that if $e$ evaluates to a value, then this value satisfies the refinement predicates that appear within the structure of $t$. We formalize this notion with a recursive definition:
\begin{align*}
\lb b \rb \,&:= \{ e \;|\; \varnothing \vdash_B e : b\}\\
\lb b\{x\col p\}\rb \,&:= 
  \{ e \;|\; (\varnothing \vdash_B e:b)
\,\wedge\, ({\rm if}\, e \many v \,{\rm then}\, p[v/x] \many {\tt true})\} \\
\lb x\bind t_x \rightarrow t\rb \,&:= 
\{ e \,|\; (\varnothing \vdash_B e : \lfloor t_x\rfloor \rightarrow \lfloor t\rfloor ) \,\wedge\,
( \forall\, v_x \in \lb t_x \rb.\, e\; v_x \in \lb t[v_x/x] \rb)\} \\
\lb \exists\, x\bind t_x .\, t\rb \,&:= 
\{ e \,|\; (\varnothing \vdash_B e : \lfloor t\rfloor ) \,\wedge\,
( \exists\, v_x \in \lb t_x \rb.\, e \in \lb t[v_x/x] \rb)\}
\end{align*}

%The addition of $\bot$ guarantees that all denotations are non-empty.
\begin{comment}
\begin{lemma}\label{non-empty}
Let $t$ be a type such that $\Gamma \vdash_w t$. Then there exists a term $e$ such that for every $\theta \in \lb \Gamma \rb$, we have $e \in \lb \theta(t)\rb$. In particular, if $\varnothing \vdash_w t$, then $\lb t \rb \neq \varnothing$.
\end{lemma}
\begin{proof} We proceed by induction on the derivation of $\Gamma \vdash_w t$(i.e. on the structure of the type $t$).

{\bf Case} {\sc WF-Base}: This follows trivially because $b$ has no free variables and for each base type, $\lb b\rb \neq \varnothing$.

{\bf Case} {\sc WF-Refn}: We have $\Gamma \vdash_w t$ and $t \equiv b\{x\col p\}$. Let $e = \bot$. Then for any $\theta \in \lb\Gamma\rb$, $\bot \in \lb b\{x \col \theta(p)\}\rb = \lb\theta(t)\rb$ by definition.

{\bf Case} {\sc WF-Func}: We have $\Gamma \vdash_w t$ and $t \equiv \functype{x}{t_x}{t'}$. By inversion we have
\begin{equation}
\Gamma \vdash_w t_x \quad {\rm and} \quad \Gamma, x\bind t_x \vdash_w t'.
\end{equation}
By the inductive hypothesis, there exists some term $e'$ such that for any $\theta' \in \lb \Gamma, x\bind t_x \rb$ we have $e' \in \lb\theta'(t')\rb$. Set $e = \lambda x.\, e'$. Let $\theta \in \lb\Gamma\rb$ and let $e_x \in \lb \theta(t_x) \rb$. Then $(\theta, x\mapsto e_x) \in \lb\Gamma, x\bind t_x\rb$ and so we have $e\; e_x$
\end{proof}
\end{comment}
\begin{comment}
We can formalize the above intuition that type denotations capture terms that diverge with the following lemma:
\begin{lemma}\label{denote-diverge}
Let $t$ be a type such that $\varnothing \vdash_w t$ and $e$ be an expression such that $\varnothing \vdash_B e : \lfloor t \rfloor$ but there does not exist any value $v$ such that $e \many v$. Then $e \in \lb t \rb$; moreover, such an $e$ always exists so in particular $\lb t \rb \neq \varnothing$.
\end{lemma}
\begin{proof}
We prove this by induction on the structure of type $t$. The cases that $t \equiv b$, a base type, or $t \equiv b\{x\col p\}$ follow trivially from the definition of $\lb t \rb$ (we can take $e = \bot$). In the next case, suppose $t \equiv \functype{x}{t_x}{t'}$. Let $e_x \in \lb t_x \rb$. We note that $e\; e_x$ cannot reduce to a value: the only rule that we can ever apply is {\sc E-App1} because $e$ is not a value and for any $e'$ such that $e \many e'$, $e'$ is also not a value. Thus we can never apply {\sc E-Prim}, {\sc E-App2}, or {\sc E-AppAbs} and no expression of the form $e'\; e_x$ is ever a value. Then by induction we have that $e\; e_x \in \lb t[e_x/x]\rb$ and so $e \in \lb\functype{x}{t_x}{t'}\rb$. To show that $\lb\functype{x}{t_x}{t'}\rb \neq \varnothing$, we use the inductive hypothesis to get an $e_x \in \lb t_x \rb$ and use it again to get a term $e' \in \lb t'[e_x/x] \rb$ that doesn't evaluate to a value. Set $e = \lambda x.\, e'$. Then $\varnothing \vdash_B e : \lfloor t_x \rfloor \rightarrow \lfloor t' \rfloor$ and for any $\hat e \in \lb t_x \rb$, either $\hat e$ doesn't evaluate to a value (in which case $e\; \hat e$ doesn't evaluate to a value either and so $e\; \hat e \in \lb t'[e_x/x]\rb$ by induction) or $\hat e \many \hat v$ for some value $\hat v$. Then $e\; \hat e \many e\; \hat v \step e'[v/x] = e' \in \lb t'[e_x/x]\rb$ and so by Lemma \ref{pres-den} $e\; \hat e \in \lb t'[e_x/x]\rb$. Therefore $e \in \lb t\rb$.

In the final case, suppose $t\equiv \existype{x}{t_x}{t'}$. By the inductive hypothesis, there exists some $e_x \in \lb t_x \rb$. Suppose that $\varnothing \vdash_B e : t'$ and that $e$ doesn't evaluate to a value. Then by induction we have that $e \in \lb t'[e_x/x]\rb$ so $e \in \lb\existype{x}{t_x}{t'}\rb$. Finally, to show that such an $e$ must exist, we have by the inductive hypothesis that there must exists some $e\in\lb t'[e_x/x]\rb$ that does not evaluate to a value, and so $e\in \lb\existype{x}{t_x}{t'}\rb$ as well.
\end{proof} 
\end{comment}

We also have the concept of the denotation of an environment $\Gamma$; we intuitively define this to be the set of all sequences of expression bindings for the variables in $\Gamma$ such that the expressions respect the denotations of the types of the corresponding variables.
A closing substitution is just a sequence of expression bindings to variables:
\[
\theta = (x_1\mapsto e_1,\,\ldots,\, x_n\mapsto e_n)
\quad {\rm with\, all}\, x_i\, {\rm distinct}
\]
We use the shorthand $\theta(x)$ to refer to $e_i$ if $x = x_i$. We define $\theta(t)$ to be the type derived from $t$ by substituting for all variables in $\theta$:
\[
\theta(t) := t[e_1/x_1]\cdots[e_n/x_n]
\]
Then we can formally define the denotation of an environment:
\[
\lb \Gamma \rb := \{ \theta = (x_1 \mapsto e_1,\ldots, x_n \mapsto e_n) \; | \;
\forall\, (x:t) \in \Gamma.\, \theta(x) \in \lb\theta(t)\rb \}.
\]

For each built-in primitive constant or function $c$ we define $ty(c)$ to include the most specific possible refinement type for $c$.
\begin{align*}
ty(\true) :=&\; \Bool\{ x : x = \true \}\\
ty({\tt false}) :=&\; \Bool\{ x : x = {\tt false}\}\\
ty(3) :=&\; \Int\{ x : x = 3\} \\
ty(n) :=&\; \Int\{ x : x = n\} \\
ty(\wedge) :=&\; 	x\bind\Bool \rightarrow y\bind\Bool \rightarrow \Bool\{ v : v = x \wedge y\}\\
ty(\neg) :=&\; x\bind\Bool \rightarrow \Bool\{ y : y = \neg x\}\\
ty(\leq) :=&\; x\bind\Int \rightarrow y\bind\Int \rightarrow \Bool\{v : v = (x \leq y)\}\\
ty(m\!\!\leq) :=&\; n\bind\Int \rightarrow \Bool\{v : v = (m \leq n)\} 
\end{align*}
and similarly for $ty(\vee)$, $ty(=)$, and $ty(m\!\!=)$. Note that we use $m\!\!\leq$ to represent an arbitrary member of the infinite family of primitives $0\!\!\leq,\, 1\!\!\leq,\, 2\!\!\leq,\ldots$. Then by the definitions above we get our primitive typing lemma:
\begin{lemma}(Primitive Typing) For every primitive $c$, 
\begin{enumerate}
\item $\varnothing \vdash c : ty(c)$ and $\varnothing \vdash_w ty(c)$.  
\item If $ty(c) = b\{x \col p\}$, then $c \in \lb ty(c) \rb$ and for all $c'$ such that $c' \neq c$, $c' \not\in \lb ty(c)\rb$.
\item If $ty(c) = \functype{x}{t_x}{t}$, then for each $v \in \lb t_x\rb$, $\delta(c,v)$ is defined and we have both $\varnothing \vdash \delta(c,v) : t[v/x] $ and $\delta(c,v) \in \lb t[v/x] \rb$. Thus $c \in \lb ty(c) \rb$.
\end{enumerate}\label{prim-typing}
\end{lemma}

\section{Meta-theory}  %%%%%%%%%%%%%%% 333333333333 %%%%%%%%%%%%%

In this section, we seek to prove the operational soundness of our language $\lambda_1$. We begin by proving several standard properties and basic facts used later on.

\begin{lemma}\label{step-determ}
The operational semantics of $\lambda_1$ are deterministic: For every expression $e$ there exists at most one term $e'$ such that $e \step e'$.
\end{lemma}

\begin{lemma}\label{weakenings}
(Weakenings of Judgments) For any environment $\Gamma$ and $x \not\in dom(\Gamma)$:
\begin{enumerate}
\item If $\Gamma \vdash e : t$ then $\Gamma, x\bind t_x \vdash e :  t$.
\item If $\Gamma \vdash s <: t$ then $\Gamma, x\bind t_x \vdash s <: t$.
\item If $\Gamma \vdash_e p$ then $\Gamma, x\bind t_x \vdash_e p$.
\item If $\Gamma \vdash_w t$ then $\Gamma, x\bind t_x \vdash_w t$.
\end{enumerate}
\end{lemma}
\begin{proof}
The proof is by mutual induction on the derivation tree of each type of judgement.

(1) {\bf TODO}

(2) {\bf TODO}

(3) {\bf TODO}

(4) {\bf TODO}
\end{proof}


\begin{lemma}\label{sub-refl}
(Reflexivity of $<:$) If $\Gamma \vdash_w t$ and $t$ is not a base type, then $\Gamma \vdash t <: t$.
\end{lemma} {\bf TODO: Write up the proof from my notes}

\begin{proof} We proceed by induction of the structure of the derivation of $\Gamma \vdash_w t$.

{\bf Case} {\sc WF-Refn}: In the base case, we have....
	
\end{proof}


Our proof of the soundness theorem begins with several helping lemmas.

\begin{lemma}{(Preservation of Denotations)
If $e \hookrightarrow^* e'$ then $e \in \lb t\rb$ iff $e' \in \lb t\rb$.}\label{pres-den}
\end{lemma}
\begin{proof}
We proceed by a case split on the definition of the denotation of a type; in other words, we use induction on the size of type $t$ (the number of arrows or existential quantifiers appearing in $t$).

First, suppose that $t \equiv b$, a base type. We appeal to the soundness of the underlying bare type system. In particular, if
$e \in \lb b \rb$ then $\varnothing \vdash_B e : b$ and so $\varnothing \vdash_B e' : b$ and $e' \in \lb b \rb$.
Conversely if $e' \in \lb b \rb$ then by determinism of the operational semantics and the typing relation, $e \in \lb b \rb$. So we see that $e \in \lb b\rb$. We use this argument implicitly in each of the other cases because we can replace $b$ with any bare type.

Second, suppose $t \equiv b\{x\col p\}$.
If $e \in \lb t\rb$ then it holds that 
%both $\varnothing \vdash_B e : t$ (in the underlying basic type system) and 
if $e \many v$ for some value v then $p[v/x] \many \true$.
%By soundness, $\varnothing \vdash_B e' : t$.
Suppose $e' \many v$ for some value $v$; then by transitive closure $e \many v$, so $p[v/x] \many \true$ and we conclude $e' \in \lb t \rb$.

If $e' \in \lb t\rb$ then it holds that %$\varnothing \vdash_B e' : t$ and 
if $e' \many v$ for some value v then $p[v/x] \many \true$. Suppose $e \many v$. We appeal to the determinism of the operational semantics: by hypothesis, $e \many e'$, so it must be the case that $e' \many v$. Then $p[v/x] \many \true$ and %by soundness of the underlying type system, $\varnothing \vdash_B e : t$. 
therefore $e \in \lb t \rb$.

Next, suppose $t \equiv x\bind t_x \rightarrow t'$.
If $e \in \lb t\rb$ then it holds that %both $\varnothing \vdash_B e : t$ and
for every $v \in \lb t_x \rb$, we have $e\, v \in \lb t'[v/x]\rb$.
Because $e \many e'$, we have $(e\, v) \many (e'\, v)$ by the operational semantics and so by induction (on the structure of  $t$) we have $e'\, v \in \lb t'[v/x]\rb$ and thus $e' \in \lb t\rb$.
If $e' \in \lb t\rb$ then it holds that %$\varnothing \vdash_B e' : t$and 
$\forall\, v \in \lb t_x \rb.\, e'\, v \in \lb t'[v/x]\rb$.
We appeal to the determinism of the operations semantics: by hypothesis, $e \many e'$, so it must be the case that $e\, v \many e'\, v$. Then by induction we have $e\, v \in \lb t'[v/x]\rb$, and thus $e\in\lb t\rb$.

Finally, suppose $t \equiv \exists\, x\bind t_x.\, t'$. We have $e \in \lb t \rb$ if and only if there exists some $v \in \lb t_x \rb$ such that 
$e \in \lb t'[v/x]\rb$. Then by induction $e \in \lb t'[v/x]\rb$ if and only if $e' \in \lb t'[v/x]\rb$. By definition, $e' \in \lb t\rb$ if and only if there exists $v \in \lb t_x\rb$ such that $e'\in\lb t'[v/x]\rb$, which completes the proof.
\end{proof}

\begin{comment}
\begin{lemma}{(Declarative Entailments) Our entailment judgement is sound with respect to the denotations of the environment: If
$\Gamma \vdash_e c$, then $\forall\, \theta.\, \theta \in \lb\Gamma\rb \Rightarrow \varnothing \vdash_e \theta(c)$.}
\label{decl-impl}
\end{lemma}

\begin{proof} %(``$\Rightarrow$" direction) 
We proceed by induction on (the length of) $\Gamma$. In the base case $\Gamma = \varnothing$, so $\theta(c) = c$ and the result is vacuous.
In the inductive case, suppose we have $\Gamma', x\bind t \vdash_e c$.
By inversion of {\sc Ent-Ext}, we must have that $t \equiv b\{x\col p\}$ and $\Gamma' \vdash_e \forall\,x\bind b.\, p \Rightarrow c$. By the inductive hypothesis, for any $\theta' \in \lb\Gamma'\rb$, $\varnothing \vdash_e \theta'(\forall\, x\bind b.\, p \Rightarrow c)$, or equivalently, $\varnothing \vdash_e \forall\, x\bind b.\, \theta'(p) \Rightarrow \theta'(c)$. By inversion of {\sc Ent-EmpI}, we have for all $\theta'' = (x\mapsto e) \in \lb b\{x\col p\}\rb$, $\varnothing \vdash_e \theta'(c)[e/x]$. Therefore, for any $\theta \in \lb \Gamma', x\bind t\rb$, if we write $\theta=(\theta', x\mapsto e)$ we have $\varnothing \vdash_e \theta(c)$.

%(``$\Leftarrow$" direction) We again proceed by induction on the length of $\Gamma$. The base case follows trivially because $\theta(c)=c$ when $\Gamma = \varnothing$. In the inductive case, suppose we have $\Gamma \equiv \Gamma', x\bind t$ and for all $\theta \in \lb \Gamma \rb$ we have $\varnothing \vdash_e \theta(c)$. 
%We can argue by induction on the derivation tree of $\varnothing \vdash_e \theta(c)$ that....
%Let $\theta \in \lb\Gamma\rb$ arbitrary and let
%Write $\theta = (\theta', x\mapsto e)$ where $e \in \lb \theta(t)\rb = \lb\theta'(t)\rb$ because $x$ cannot appear free in $t$. 
\end{proof}
\end{comment}

\begin{lemma}{(Type Denotations) Our typing and subtyping relations are sound with respect to the denotational semantics of our types:\\
1. If $\Gamma \vdash t_1 <: t_2$ then $\forall \theta. \theta \in \lb \Gamma \rb \Rightarrow \lb\theta(t_1)\rb \subseteq \lb\theta(t_2)\rb$.\\
2. If $\Gamma \vdash e : t$ then $\forall \theta. \theta \in \lb \Gamma \rb \Rightarrow \theta(e) \in \lb\theta(t)\rb.$
}\label{type-denote}\\
3. TODO: Do I need to prove that well-formedness is sound too?
\end{lemma}

The proof is by mutual induction on the derivation trees of the respective subtyping and typing judgements. The need for mutual induction contrasts with Lemma 4 of [VSJ$^+$14] and comes from the appearance of the typing judgement $\Gamma \vdash e_x : t_x $ in the antecedent of rule {\sc S-Witn}.
     
\begin{proof} 
(1) Suppose $\Gamma \vdash t_1 <: t_2$. We proceed by induction on the derivation tree of the subtyping relation.

{\bf Case} $\textsc{Sub-Base}$: We have that 
$\Gamma \vdash b\{x_1\col p_1\} <: b\{x_2\col p_2\}$ where $t_1 \equiv b\{x_1\col p_1\}$ and $t_2 \equiv b\{x_2\col p_2\}$.
By inversion, 
\[\Gamma; x_1\bind b\{x_1\col p_1\} \vdash_e  p_2[x_1/x_2].\] 
By inversion of {\sc Ent-Ext} we have 
\begin{equation}\Gamma \vdash_e \forall\, x_1\bind b.\, p_1 \Rightarrow p_2[x_1/x_2]
.\end{equation}
By Lemma \ref{decl-impl} we have
\[
\forall\,\theta.\,\theta\in\lb\Gamma\rb \Rightarrow 
\varnothing \vdash_e \theta(\forall\, x_1\bind b.\, p_1 \Rightarrow p_2[x_1/x_2])
,\]
or equivalently
\begin{equation}
\forall\,\theta.\,\theta\in\lb\Gamma\rb \Rightarrow 
\varnothing \vdash_e \forall\, x_1\bind b.\, \theta(p_1) \Rightarrow \theta(p_2)[x_1/x_2]
\end{equation}
By inversion of rule {\sc Ent-EmpI}, we have
\begin{equation}
\forall\,\theta.\,\theta\in\lb\Gamma\rb \Rightarrow 
\forall\, (x_1\mapsto e)\in \lb b\{x_1\col \theta(p_1)\}\rb \Rightarrow \varnothing \vdash_e \theta(p_2)[x_1/x_2][e/x_1].
\end{equation}
Then by inversion of rule {\sc Ent-EmpP}, we obtain
\begin{equation}\label{3.1.0}
\forall\,\theta.\,\theta\in\lb\Gamma\rb \Rightarrow 
\forall\, (x_1\mapsto e)\in \lb b\{x_1\col \theta(p_1)\}\rb \Rightarrow \theta(p_2)[e/x_2] \many \true.
\end{equation}
We need to show $\forall \theta.\; 
\theta \in \lb \Gamma \rb \Rightarrow 
\lb\theta(b\{x_1:p_1\})\rb \subseteq \lb\theta(b\{x_2:p_2\})\rb.$
Equivalently,
\begin{align}\label{3.1.1}
\forall\theta.\,\theta\in\lb\Gamma\rb \Rightarrow&
\{ e \,|\, \varnothing \vdash_B e:b \;\wedge\; 
  ({\rm if}\, e \many v \,{\rm then}\, \theta(p_1[v/x_1]) \many \true)\}\\
\subseteq &\{ e \,|\, \varnothing \vdash_B e:b \;\wedge\; 
  ({\rm if}\, e \many v \,{\rm then}\, \theta(p_2[v/x_2]) \many \true)\}\label{3.1.2}
\end{align}
Let $\theta \in \lb\Gamma\rb$ be a closing substitution and
let $e$ a term in set (\ref{3.1.1}), and suppose $e \many v$. Then $\theta(p_1[v/x_1]) \many \true$, and so $\theta' = (\theta, x_1 \mapsto v) \in \lb\Gamma, b\{ x_1\col \theta(p_1)\}\rb$.
From (\ref{3.1.0}) we have
$\theta(p_2[v/x_2]) = \theta(p_2)[v/x_2] \many \true$,
and so $e$ lies in set (\ref{3.1.2}), thus proving the desired containment.

{\bf Case} $\textsc{Sub-Fun}$: We have that
$\Gamma \vdash x_1\bind s_1 \rightarrow t'_1 <: x_2\bind s_2 \rightarrow t'_2$ where $t_1 \equiv x_1\bind s_1 \rightarrow t'_1$ and $t_2 \equiv x_2\bind s_2 \rightarrow t'_2$. By inversion of this rule,
\[
\Gamma \vdash s_2 <: s_1 \;\;\;\;{\rm and}\;\;\;\;
\Gamma,x_2\bind s_2 \vdash t'_1[x_2/x_1] <: t'_2
\]
By the inductive hypothesis,
\[
\forall\theta.\, \theta \in \lb\Gamma\rb \Rightarrow
\lb\theta(s_2)\rb \subseteq\lb\theta(s_1)\rb 
\]
and
\begin{equation}\label{3.2.0}
\forall\theta.\, \theta \in \lb\Gamma, x_2\bind s_2\rb \Rightarrow
\lb\theta(t'_1[x_2/x_1])\rb \subseteq\lb\theta(t'_2)\rb 
\end{equation}
We need to show $\forall \theta.\; 
\theta \in \lb \Gamma \rb \Rightarrow 
\lb\theta(x_1\bind s_1\rightarrow t'_1)\rb \subseteq \lb\theta(x_2\bind s_2\rightarrow t'_2)\rb.$
Equivalently,
\begin{align} \label{3.2.1}
\forall\theta.\,\theta\in\lb\Gamma\rb \Rightarrow&
\{ e \,|\, \varnothing \vdash_B e:\lfloor s_1\rfloor \rightarrow \lfloor t'_1\rfloor \;\wedge\; 
  (\forall\, e' \in \lb \theta(s_1)\rb.\, e\,e' \in\lb \theta(t'_1[e'/x_1])\rb)\}\\
\subseteq &\{ e \,|\, \varnothing \vdash_B e:\lfloor s_2\rfloor \rightarrow \lfloor t'_2\rfloor \;\wedge\; 
  (\forall\, e' \in \lb \theta(s_2)\rb.\, e\,e' \in\lb \theta(t'_2[e'/x_2])\rb)\}\label{3.2.2}
\end{align}
Fix $\theta \in \lb\Gamma\rb$ and let $e$ be a term in set (\ref{3.2.1}) and let $e' \in \lb \theta(s_2)\rb$. Then by induction, $e' \in \lb \theta(s_1)\rb$. So $(e\, e') \in \lb(\theta(t'_1[e'/x_1])\rb$. Let $\theta' = (\theta, x_2 \mapsto e')$. From (\ref{3.2.0}) we also have that 
$\lb\theta'(t'_1[x_2/x_1])\rb \subseteq \lb\theta'(t'_2)\rb$.
But $\theta'(t'_1[x_2/x_1]) = \theta(t'_1[x_2/x_1])[e'/x_2] = \theta(t'_1[e'/x_1])$ and $\theta'(t'_2) = \theta(t'_2)[e'/x_2]$.
Therefore  $(e\; e') \in \lb \theta(t'_1[e'/x_2])\rb \subseteq \lb \theta(t'_2[e'/x_2])\rb$ and so $e$ is in set (\ref{3.2.2}) as desired.

{\bf Case} $\textsc{Sub-Witn}$: We have that
$\Gamma \vdash t_1 <: \exists\, x\bind t_x.\, t'_2$ where $t_2 \equiv \exists\, x\bind t_x.\, t'_2$. By inversion, there exists some term $e_x$ such that
\[
\Gamma \vdash e_x : t_x \quad {\rm and} \quad \Gamma \vdash t_1 <: t'_2[e_x/x].
\]
By the inductive hypothesis, we have
\begin{equation}\label{3.3.1}
\forall\, \theta.\, \theta\in \lb\Gamma\rb \Rightarrow \lb \theta(t_1) \rb \subseteq \lb \theta(t'_2[e_x/x]) \rb 
\end{equation}
and by mutual induction we also have
\[
\forall\, \theta.\, \theta\in \lb\Gamma\rb \Rightarrow \theta(e_x) \in \lb\theta(t_x)\rb.
\]
We need to show that $\forall\, \theta$, if $\theta \in \lb\Gamma\rb$, then $\lb\theta(t_1)\rb \subseteq \lb \theta(\exists\, x\bind t_x.\, t'_2)\rb$. Fix some $\theta \in \lb\Gamma\rb.$ Then
\begin{equation} \label{3.3.2}
\lb \theta(\exists\, x\bind t_x.\, t'_2)\rb
= \{ e \,|\, \varnothing \vdash_B e : \lfloor \theta(t'_2)\rfloor \;\wedge\; 
  (\exists\, e' \in \lb \theta(t_x)\rb.\, e \in\lb \theta(t'_2)[e'/x]\rb)\}
\end{equation}
because $\theta(\exists\, x\bind t_x.\, t'_2) = \exists\, x\bind\theta(t_x).\, \theta(t'_2).$
Let $e \in \lb\theta(t_1)\rb$ and set $e' = \theta(e_x) \in \lb\theta(t_x)\rb.$ Then by (\ref{3.3.1}), $e \in \lb\theta(t'_2[e_x/x])\rb$ and by definition of the denotation of a type, in every case $\varnothing \vdash_B e : \lfloor \theta(t'_2[e_x/x])\rfloor = \lfloor\theta(t'_2)\rfloor$. We conclude by noting $\theta(t'_2[e_x/x]) = \theta(t'_2)[e'/x]$ and so $e$ is in the right hand side of (\ref{3.3.2}).

{\bf Case} $\textsc{Sub-Bind}$: We have that $\Gamma \vdash \exists\, x\bind t_x.\, t'_1 <: t_2$ where $t_1 \equiv \exists\, x\bind t_x.\, t'_1$. By inversion we have
\[
\Gamma, x\bind t_x\vdash t'_1 <: t_2 \quad {\rm and}\quad x \not\in free(t_2).
\]
By the inductive hypothesis, we have
\begin{equation}
\forall\,\theta.\, \theta\in\lb\Gamma,x\bind t_x\rb \Rightarrow \lb\theta(t'_1)\rb \subseteq \lb\theta(t_2)\rb.
\label{3.4.1}	
\end{equation}
We need to show that for every $\theta \in \lb\Gamma\rb$ that it holds that $\lb \theta(\exists\, x\bind t_x.\, t'_1)\rb \subseteq \lb\theta(t_2)\rb$. Fix some $\theta \in \lb\Gamma\rb$ and let $e \in \lb \theta(\exists\, x\bind t_x.\, t'_1)\rb$. By definition, $\theta(\exists\, x\bind t_x.\, t'_1) = \exists\, x\bind \theta(t_x).\, \theta(t'_1)$ so
\begin{equation}\label{3.4.2}
\lb \theta(\existype{x}{t_x}{t'_1})\rb = \{ e \,|\, \varnothing \vdash_B e: \lfloor\theta(t'_1)\rfloor \;\wedge\; (\exists\, e' \in \lb\theta(t_x)\rb.\, e \in \lb\theta(t'_1)[e'/x]\rb)\}.
\end{equation}
Take $e'$ as in (\ref{3.4.2}) and let $\theta' = (\theta, x\mapsto e')$. We note that $\theta' \in \lb\Gamma,x\bind t_x\rb$ because $\theta'(x) = e' \in \lb\theta(t_x)\rb = \lb\theta'(t_x)\rb$ where the last equality follows from the fact that $x$ cannot appear free in $t_x$. Then $e \in \lb\theta'(t'_1)\rb$, so from (\ref{3.4.1}) we can conclude $e \in \lb\theta'(t_2)\rb = \lb\theta(t_2)\rb$ because $x$ does not appear free in $t_2$ so $\theta'(t_2)=\theta(t_2)$. \\

(2) Suppose $\Gamma \vdash e : t$. We proceed by induction on the derivation tree of the typing relation.

{\bf Case} {\sc T-Prim}: We have $\Gamma \vdash e : t$ where $e \equiv c$, a built-in primitive function or constant. By inversion, $ty(c) = t$. Let $\theta \in \lb \Gamma \rb$.
In one case $t \equiv b\{x\col p\}$; then by Lemma \ref{prim-typing} on constants, $\theta(c) = c \in \lb ty(c)\rb = \lb \theta(ty(c))\rb$. In the other case, $ty(c) \equiv \functype{x}{t_x}{t'}$; by Lemma \ref{prim-typing},
$\delta(c,v) \in \lb t'[v/x]\rb$ for any $v \in \lb t_x\rb.$ There are no free variables in $c$ or $t$ so again $\theta(c) \in \lb\theta(ty(c))\rb$.

{\bf Case} {\sc T-Var}: We have $\Gamma \vdash e : t$ where $e \equiv x$. By inversion, $(x\bind t) \in \Gamma$. Then for any $\theta \in \lb\Gamma\rb,$ we have by definition $\theta(x) \in \lb \theta(t)\rb$ as desired.

{\bf Case} {\sc T-Abs}: We have $\Gamma \vdash e : t$ where $e \equiv \lambda x.e'$ and $t \equiv \functype{x}{t_x}{t'}$. By inversion, $\Gamma,x\bind t_x \vdash e' : t'$ and by the inductive hypothesis,
\begin{equation}\label{3.7.1}
\forall\theta'.\, \theta' \in \lb\Gamma,x\bind t_x\rb \Rightarrow \theta'(e') \in \lb\theta'(t')\rb.\end{equation}
Let $\theta \in \lb\Gamma\rb$ and let $e_x \in \lb \theta(t_x)\rb$. Then let
\[\theta' := (\theta,x\mapsto e_x) \in \lb\Gamma,x\bind t_x\rb.
\] 
%because we chose $\theta(x) = \theta(v) \in \lb\theta(t_1)\rb$.
Then from (\ref{3.7.1}), 
\begin{equation}\label{3.7.2}
\theta(e')[e_x/x] = \theta'(e') \in \lb\theta'(t')\rb = \lb\theta(t')[e_x/x]\rb.
\end{equation}

We need to show that for every $\theta \in \lb\Gamma\rb$, it holds that 
\begin{align*}
\theta(e) \in \lb \theta(&\functype{x}{t_x}{t'}) \rb = \lb \functype{x}{\theta(t_x)}{\theta(t')}\rb\\
=&\; \{ \hat{e} \;|\; (\varnothing \vdash_B {\hat e} : \lfloor t_x \rfloor \rightarrow \lfloor t' \rfloor) \wedge ( \forall\, \hat e_x \in \lb \theta(t_x) \rb.\; \hat e \; \hat e_x \in \lb\theta(t')[\hat e_x/x] \rb) \}
\end{align*}
We have $\varnothing \vdash_B \theta(e) : \lfloor t_x \rfloor \rightarrow \lfloor t' \rfloor$ because substitutions do not affect bare types, only the refinement predicates.
We have
$\theta(e)\; e_x = (\lambda x.\theta(e'))\; e_x.$
There are two cases for $e_x$: first, if $e_x$ does not evaluate to any value then the only reduction rules we can ever apply are {\sc E-App1} and {\sc E-App2} and so $\theta(e)\; e_x$ can never evaluate to a value. By Lemma \ref{denote-diverge} we have that $\theta(e)\; e_x \in \lb t'[e_x/x]\rb$. In the other case suppose that there exists some value such that $e_x \many v_x$. Then 
$\theta(e)\; e_x = (\lambda x.\theta(e'))\; e_x \many (\lambda x.\theta(e'))\; v_x \step \theta(e')[v_x/x] \in \lb \theta(t')[v_x/x]\rb$. 
By a forthcoming lemma (TODO: this), $\lb\theta(t')[v_x/x]\rb \subseteq \lb\theta(t')[e_x/x]\rb$. Then by Lemma \ref{denote-diverge} we conclude that $\theta(e)\; e_x \in \lb\theta(t')[e_x/x]\rb$ and so $\theta(e) \in \lb\theta(t)\rb$.

{\bf Case} {\sc T-App}: We have $\Gamma \vdash e : t$ where $e \equiv e'\; e_x$ and $t \equiv \existype{x}{t_x}{t'}$. By inversion,
$\Gamma \vdash e' : \functype{x}{t_x}{t'}$ and $\Gamma \vdash e_x : t_x$. 
By the inductive hypothesis we have
\begin{equation}\label{3.8.1}
\forall \theta.\, \theta\in\lb\Gamma\rb \Rightarrow 
\theta(e') \in \lb\theta(\functype{x}{t_x}{t'})\rb\end{equation}
and
\begin{equation}\label{3.8.2}
\forall \theta.\, \theta \in \lb\Gamma\rb \Rightarrow
\theta(e_x) \in \lb\theta(t_x)\rb.
\end{equation}
From (\ref{3.8.1}), we have that for all $\theta \in \lb\Gamma\rb$, $\theta(e')\; \theta(e_x) \in \lb\theta(t')[\theta(e_x)/x]\rb$. Thus
\begin{equation}
\theta(e) = \theta(e')\; \theta(e_x) \in \lb \existype{x}{\theta(t_x)}{\theta(t')}\rb = \lb\theta(\existype{x}{t_x}{t'})\rb
.\end{equation}

{\bf Case} {\sc T-Let}: We have $\Gamma \vdash e : t$ where
$e \equiv \letin{x}{e_x}{e'}$.
By inversion, we have $\Gamma \vdash e_x : t_x$,\; $(\Gamma,x\bind t_x) \vdash e' : t$, and $\Gamma \vdash_w t$
for some $t_x$. Then by the inductive hypothesis we have
\[
\foralltheta \Rightarrow \theta(e_x) \in \lb\theta(t_x)\rb
\] and 
\begin{equation}\label{3.9.1}
\forall\theta'.\, \theta' \in \lb\Gamma,x\bind t_x\rb \Rightarrow \theta'(e') \in \lb\theta'(t)\rb.\end{equation}
Let $\theta \in \lb\Gamma\rb$. There are two cases for the semantics of $e_x$. In the case that there exists some value $v_x$ such that $\theta(e_x) \many v_x$, let $\theta' = (\theta, x\mapsto v_x)\in \lb\Gamma,x\bind t_x\rb$ because we chose $\theta'(x) = v_x \in \lb\theta(t_x)\rb$. 
From the operational semantics $\theta(\letin{x}{e_x}{e'}) = \letin{x}{\theta(e_x)}{\theta(e')} \many \letin{x}{v_x}{\theta(e')} \step \theta(e')[v_x/x]$. 
Then from (\ref{3.9.1}),
\[
\theta(e')[v_x/x] = \theta'(e') \in \lb\theta'(t)\rb = \lb\theta(t)[v_x/x]\rb = \lb\theta(t)\rb,
\]
where the last equality follows from the fact that the judgement $\Gamma \vdash_w t$ implies $\varnothing \vdash_w \theta(t)$ by part 3 of this lemma, which in turn implies that $x$ cannot be free in $\theta(t)$.
The above implies
$\theta(e) \many \theta(e')[v_x/x]$, so by Lemma \ref{pres-den}, $\theta(e) \in \lb\theta(t)\rb$.

In the second case, $\theta(e_x)$ does not reduce to any value. In that case, the only rule we can ever apply to $\theta(e) = \letin{x}{\theta(e_x)}{\theta(e')}$ is {\sc E-Let} so $\theta(e)$ never reduces to a value, and by Lemma \ref{denote-diverge} $\theta(e) \in \lb\theta(t)\rb$.

{\bf Case} {\sc T-Ann}: We have $\Gamma \vdash e : t$ where $e \equiv (e'\col t)$. By inversion, $\Gamma \vdash e' : t$ and by the inductive hypothesis, $\theta(e') \in \lb\theta(t)\rb$. By the operational semantics of type annotations, 
$\theta(e) = (\theta(e')\col\theta(t)) \step \theta(e') \in \lb\theta(t)\rb$, so we conclude that $\theta(e) \in \lb\theta(t)\rb$ by Lemma \ref{pres-den}.

{\bf Case} {\sc T-Sub}: We have $\Gamma \vdash e : t $ and by inversion, we have $\Gamma \vdash e : s$ and $\Gamma \vdash s <: t$ for some type $s$. 
By the inductive hypothesis, $\foralltheta \Rightarrow \theta(e) \in \lb\theta(s)\rb$ and by mutual induction, part 1 of the Lemma gives us that  $\foralltheta \Rightarrow \lb\theta(s)\rb \subseteq \lb\theta(t)\rb$. Then we conclude that $\foralltheta \Rightarrow \theta(e) \in \lb\theta(t)\rb$.

{\bf Case} {\sc T-Bot}: We have $\Gamma \vdash \bot : t$ where $t \equiv b\{x \col p\}$. Then because $\bot$ cannot evaluate, we have that $\bot \in \lb b\{x\col \theta(p)\}\rb = \lb \theta(t)\rb$.
\end{proof}

\begin{proof}
(1) Suppose $\Gamma \vdash e_x:t_x$ and $\Gamma, x\bind t_x ,\Gamma' \vdash t_1 <: t_2$. We proceed by mutual induction on the derivation tree of the subtyping relation.

{\bf Case} $\textsc{Sub-Base}$: We have that 
$\Gamma,x\bind t_x,\Gamma' \vdash b\{x_1\col p_1\} <: b\{x_2\col p_2\}$ where $t_1 \equiv b\{x_1\col p_1\}$ and $t_2 \equiv b\{x_2\col p_2\}$.
By inversion, 
\[\Gamma,x\bind t_x,\Gamma',x_1\bind b\{x_1\col p_1\} \vdash p_2[x_1/x_2].\] 
By inversion of {\sc Ent-Ext} we have 
\begin{equation}\label{311}
\Gamma,x\bind t_x,\Gamma' \vdash \forall x_1\bind b.\; p_1 \Rightarrow p_2[x_1/x_2].\end{equation}
By Lemma \ref{decl-impl}, we have
\begin{align}
\forall\, \theta^*.\, \theta^* \in \lb \Gamma,x\bind t_x,\Gamma'\rb \Rightarrow \varnothing \;&\vdash_e \theta^*(\forall x_1\bind b.\; p_1 \Rightarrow p_2[x_1/x_2]) \\
&= \forall\, x_1\bind b.\, \theta^*(p_1) \Rightarrow \theta^*(p_2)[x_1/x_2].
\end{align}

Let $e_x$ be an expression such that $\Gamma \vdash e_x : t_x$.
The above is equivalent to:
\[
\forall\, \theta, \theta'.\, (\theta, x \mapsto \theta(e_x), \theta') \in \lb\Gamma, x\bind t_x, \Gamma'\rb \Rightarrow \big( \varnothing \vdash_e
\forall\, x_1\bind b.\, \theta'(\theta(p_1)[\theta(e_x)/x]) \Rightarrow \theta'(\theta(p_2)[\theta(e_x)/x]))[x_1/x_2] \big)
\]
or equivalently,
\begin{equation}
\forall\, \theta, \theta'.\, (\theta, \theta') \in \lb\Gamma, \Gamma'[e_x/x]\rb \Rightarrow \big(\varnothing \vdash_e
\forall\, x_1\bind b.\, (\theta, \theta')(p_1[e_x/x]) \Rightarrow (\theta,\theta')(p_2[e_x/x][x_1/x_2])\big)
\end{equation}

TODO: can't finish this one yet.
\begin{comment}
The validity of the implication in (\ref{311}) means that
\begin{equation}\label{312}
\forall\theta^*. \theta^*\in\lb\Gamma,x:t_x,\Gamma',v_1:b\rb \Rightarrow
(\theta^*(p_1)\many\true)\Rightarrow(\theta^*(p_2[v_1/v_2])\many\true)
\end{equation}

Let $(\theta,\theta',v_1\mapsto e_1) \in \lb\Gamma,\Gamma'[e_x/x],v_1:b\rb$ be a closing substitution. Then
\[(\theta,x\mapsto e_x,\theta',v_1\mapsto e_1)
\in \lb \Gamma,x:t_x,\Gamma',v_1:b\rb
\]
because for each $(y:t_y[e_x/x]) \in \Gamma'[e_x/x]$ we have
$\theta'(y) \in\lb\theta'(t_y[e_x/x])\rb$ which implies
$(x\mapsto e_x,\theta')(y) \in \lb(x\mapsto e_x,\theta')(t_y)\rb$.

Suppose it were the case that $(\theta,\theta',v_1\mapsto e_1)(p_1[e_x/x]) \many \true$. Then 
\[
(\theta,x\mapsto e_x,\theta',v_1\mapsto e_1)
(p_1) \many \true
\] and so by (\ref{312}) \[
(\theta,x\mapsto e_x,\theta',v_1\mapsto e_1)
(p_2[v_1/v_2]) \many \true
\]
which in turn implies
\[
(\theta,\theta',v_1\mapsto e_1)((p_2[v_1/v_2])[e_x/x]) \many\true.
\]
Using the fact that $(p_2[v_1/v_2])[e_x/x] = (p_2[e_x/x])[v_1/v_2]$,
this gives us the entailment 
\begin{equation}\label{313}
\Gamma,\Gamma'[e_x/x] \vdash \forall v_1:b.\, 
  p_1[e_x/x] \Rightarrow (p_2[e_x/x])[v_1/v_2]
\end{equation}
By {\sc Ent-Ext},
\[
\Gamma,\Gamma'[e_x/x],v_1:b\{p_1[e_x/x]\} \vdash (p_2[e_x/x])[v_1/v_2].
\]
By {\sc Sub-Base},
\[
\Gamma,\Gamma'[e_x/x] \vdash b\{v_1 : p_1[e_x/x]\} <: b\{ v_2 : p_2[e_x/x]\}
\]
We know $t_1[e_x/x] = b\{v_1 : p_1[e_x/x]\}$ and likewise $t_2[e_x/x] = b\{v_2:p_2[e_x/x]\}$.
Therefore, we conclude that 
$\Gamma,\Gamma'[e_x/x] \vdash t_1[e_x/x] <: t_2[e_x/x].$
\end{comment}
\begin{comment}

{\bf Case} $\textsc{Sub-Fun}$: We have that
$\Gamma, x:t_x,\Gamma' \vdash x_1:s_1 \rightarrow t'_1 <: x_2:s_2 \rightarrow t'_2$ where $t_1 \equiv x_1:s_1 \rightarrow t'_1$ and $t_2 \equiv x_2:s_2 \rightarrow t'_2$. By inversion
\[
\Gamma,x:t_x,\Gamma' \vdash s_2 <: s_1 \;\;\;\;{\rm and}\;\;\;\;
\Gamma,x:t_x,\Gamma',x_2:s_2 \vdash t'_1[x_2/x_1] <: t'_2
\]
Applying the inductive hypothesis to the above, we get
\begin{equation}\label{321}
\Gamma,\Gamma'[e_x/x] \vdash s_2[e_x/x] <: s_1[e_x/x]
\end{equation} 
and
\begin{equation}\label{322}
\Gamma,\Gamma'[e_x/x],x_2:s_2[e_x/x] \vdash (t'_1[x_2/x_1])[e_x/x] <: t'_2[e_x/x]
\end{equation}
We necessarily have that $x \neq x_1$ so
$(t'_1[x_2/x_1])[e_x/x] = (t'_1[e_x/x])[x_2/x_1]$.
By rule {\sc Sub-Fun} applied to (\ref{321}) and (\ref{322}),
\[
\Gamma,\Gamma'[e_x/x] \vdash x_1:s_1[e_x/x] \rightarrow t'_1[e_x/x] <: x_2:s_2[e_x/x] \rightarrow t'_2[e_x/x]
\]
This is the same as 
$\Gamma,\Gamma'[e_x/x] \vdash t_1[e_x/x] <: t_2[e_x/x]$.

(2) Suppose $\Gamma \vdash e_x:t_x$ and $\Gamma, x:t_x ,\Gamma' \vdash e : t$. We proceed by induction on the derivation tree of the typing judgment $e:t$.

{\bf Case} {\sc Syn-Var}: We have $\Gamma, x:t_x,\Gamma' \vdash e : t$ where $e \equiv y$. By inversion we have $(\Gamma,x:t_x,\Gamma')(y) = t$. There are three possibilities for where in the environment $y:t$ is bound.
First, suppose $\Gamma(y) = t$. Then, necessarily, $y\neq x$ and $y[e_x/x] = y.$ But $x:t_x$ is bound to the right of $\Gamma$, so $x$ cannot appear in $t$ and $t = t[e_x/x]$. By rule {\sc Syn-Var} we have $\Gamma, \Gamma'[e_x/x] \vdash y : t$ and so
$\Gamma, \Gamma'[e_x/x] \vdash y[e_x/x] : t[e_x/x] $.

Next suppose $y \equiv x$. Then $t \equiv t_x$. Also, $x:t_x$ is bound to the right of $\Gamma$, so $x$ cannot appear in $t_x$ (i.e. $x$ cannot be free in its own type). So $t_x = t_x[e_x/x] = t[e_x/x]$.
We also have $e_x = x[e_x/x] = y[e_x/x]$ and 
By hypothesis, $\Gamma \vdash e_x : t_x$ and this judgment remains true with respect to more bindings on variables that don't appear in $e_x$ or $t_x$; so $\Gamma,\Gamma'[e_x/x] \vdash e_x : t_x$. By the above equalities we see $\Gamma,\Gamma'[e_x/x] \vdash y[e_x/x] : t[e_x/x]$.

Finally, suppose $\Gamma'(y) = t$. 
Then $\Gamma'[e_x/x](y) = t[e_x/x]$. 
By rule {\sc Syn-Var} we have $\Gamma,\Gamma'[e_x/x] \vdash y : t[e_x/x]$. We necessarily have that $y\neq x$ and so $y[e_x/x] = y$. Thus we conclude
$\Gamma,\Gamma'[e_x/x] \vdash y[e_x/x] : t[e_x/x]$.


{\bf Case} {\sc Syn-Con}:We have $\Gamma, x:t_x,\Gamma' \vdash e : t$ where $e \equiv c$. By inversion, $t = {\sf prim}(c)$. By our assumptions on constants, neither $c$ nor ${\sf prim}(c)$ contain free variables so $c[e_x/x] = c$ and ${\sf prim}(c)[e_x/x]$.
By rule {\sc Syn-Con},
$\Gamma,\Gamma'[e_x/x] \vdash c : t$ and so
$\Gamma,\Gamma'[e_x/x] \vdash c[e_x/x] : t[e_x/x]$
because the environment can be artibrary.

{\bf Case} {\sc Syn-Ann}:We have $\Gamma, x:t_x,\Gamma' \vdash e : t$ where $e \equiv (e':t)$. By inversion, we have $\Gamma, x:t_x,\Gamma' \vdash e' : t$ and by the induction hypothesis, 
$\Gamma, \Gamma'[e_x/x] \vdash e'[e_x/x] : t[e_x/x]$. By rule {\sc Syn-Ann}, we get
\begin{equation}\label{351}
\Gamma, \Gamma'[e_x/x] \vdash (e'[e_x/x] : t[e_x/x]) : t[e_x/x]
\end{equation}
By definition of substitutions $(e'[e_x/x] : t[e_x/x]) = (e':t)[e_x/x] = e[e_x/x]$, so from (\ref{351}) we immediately get $\Gamma, \Gamma'[e_x/x] \vdash e[e_x/x] : t[e_x/x]$

{\bf Case} {\sc Syn-App}:We have $\Gamma, x:t_x,\Gamma' \vdash e : t$ where $e \equiv e'\, v$ and $t \equiv t'[v/y]$ for some value $v$ (per the syntax). By inversion,
$\Gamma, x:t_x, \Gamma' \vdash e' : (y:s'\rightarrow t')$
and $\Gamma, x:t_x, \Gamma' \vdash v : s'$.
By the inductive hypothesis,
\begin{equation}\label{361}
\Gamma,\Gamma'[e_x/x] \vdash e'[e_x/x] : (y:s'[e_x/x]\rightarrow t'[e_x/x])
\end{equation}
and
\begin{equation}\label{362}
\Gamma,\Gamma'[e_x/x] \vdash v[e_x/x] : s'[e_x/x.]
\end{equation}
By rule {\sc Syn-App}
\begin{equation}
\Gamma,\Gamma'[e_x/x] \vdash e'[e_x/x]\, v[e_x/x] : (t'[e_x/x])[v[e_x/x]/y]
\end{equation}
Now by the definition of substitutions we have
$e'[e_x/x]\, v[e_x/x] = (e'\;v)[e_x/x] \equiv e[e_x/x]$ and 
$(t'[e_x/x])[v[e_x/x]/y] = (t'[v/y])[e_x/x] \equiv t[e_x/x]$.
Therefore, we conclude
$\Gamma,\Gamma'[e_x/x] \vdash e[e_x/x] : t[e_x/x]$.

{\bf Case} {\sc Chk-Syn}:We have $\Gamma, x:t_x,\Gamma' \vdash e : t$. By inversion, we have $\Gamma,x:t_x,\Gamma' \vdash e : s$
and $\Gamma, x:t_x, \Gamma' \vdash s <: t$ for some type $s$. By the inductive hypothesis we have
\begin{equation}
\label{371}
\Gamma,\Gamma'[e_x/x] \vdash e[e_x/x] : s[e_x/x]
\end{equation}
and by part (1) of the Lemma we have
\begin{equation}
\label{372}
\Gamma,\Gamma'[e_x/x] \vdash s[e_x/x] <: t[e_x/x].
\end{equation}
Then by rule {\sc Chk-Syn} we have
$\Gamma,\Gamma'[e_x/x] \vdash e[e_x/x] : t[e_x/x]$.

{\bf Case} {\sc Chk-Lam}:We have $\Gamma, x:t_x,\Gamma' \vdash e : t$ where $e \equiv \lambda y. e'$ 
and $t \equiv y:t_1 \rightarrow t_2$. By inversion,
$\Gamma, x:t_x,\Gamma',y:t_1 \vdash e' : t_2$. By the inductive hypothesis
\begin{equation}
\Gamma,\Gamma'[e_x/x],y:t_1[e_x/x] \vdash e'[e_x/x] : t_2[e_x/x].
\end{equation}
Then by rule {\sc Chk-Lam}
\begin{equation}
\Gamma,\Gamma'[e_x/x] \vdash \lambda y.(e'[e_x/x]) : (y:t_1[e_x/x] \rightarrow t_2[e_x/x]).
\end{equation}
By definition of substitution, we can rewrite the above as
\[
\Gamma,\Gamma'[e_x/x] \vdash (\lambda y.e')[e_x/x] : (y:t_1 \rightarrow t_2)[e_x/x].
\]

{\bf Case} {\sc Chk-Let}:We have $\Gamma, x:t_x,\Gamma' \vdash e : t$ where $e \equiv (\letin{y}{e_1}{e_2})$ and $t \equiv t_2$. By inversion, $\Gamma,x:t_x,\Gamma' \vdash e_1 : t_1$ and 
$\Gamma,x:t_x,\Gamma',y:t_1 \vdash e_2 : t_2$ for some type $t_1$. By the inductive hypothesis we have
\begin{equation}
\Gamma,\Gamma'[e_x/x] \vdash e_1[e_x/x] : t_1[e_x/x]
\end{equation}and
\begin{equation}
\Gamma,\Gamma'[e_x/x], y:t_1[e_x/x] \vdash e_2[e_x/x] : t_2[e_x/x]
\end{equation}
Then by rule {\sc Chk-Let},
\begin{equation}
\Gamma,\Gamma'[e_x/x] \vdash \letin{y}{e_1[e_x/x]}{e_2[e_x/x]} : t_2[e_x/x]
\end{equation}
which we can write as
\[
\Gamma,\Gamma'[e_x/x] \vdash (\letin{y}{e_1}{e_2})[e_x/x]:t_2[e_x/x]
\] to complete the proof of the final case.
\end{comment}
\end{proof}





























\newpage {\sc All the recent work goes here} 


\begin{lemma}(The Substitution Lemma) If $\Gamma \vdash e_x : t_x$ then\\
1. If $\Gamma, x\bind t_x, \Gamma' \vdash t_1 <: t_2$ then
\[
\Gamma, \Gamma'[e_x/x] \vdash t_1[e_x/x] <: t_2[e_x/x].
\]
2. If $\Gamma, x\bind t_x, \Gamma' \vdash e : t$ then
\[
\Gamma, \Gamma'[e_x/x] \vdash e[e_x/x] : t[e_x/x].
\]
3. If $\Gamma, x\bind t_x, \Gamma' \vdash_w t$ then
\[
\Gamma, \Gamma'[e_x/x] \vdash_w t[e_x/x]
\]
\end{lemma}

\begin{proof}
We proceed by mutual induction 	
	
	
	
	
\end{proof}

\begin{lemma}\label{types-wf}
(Well-formedness of types in judgements) 
If $\Gamma \vdash e: t$ and $\vdash_{w} \Gamma$ then $\Gamma \vdash_w t$.
\end{lemma}

\begin{proof} 
We proceed by induction on the derivation tree of the judgment $\varnothing \vdash e : t$.

{\bf Case} {\sc T-Prim}: We have $e \equiv c$. By inversion, $t = ty(c)$ and by Lemma \ref{prim-typing} we have $\varnothing \vdash_ w ty(c)$. By repeated application of Lemma \ref{weakenings}, we have $\Gamma \vdash_w ty(c)$. 

{\bf Case} {\sc T-Var}: We have $\Gamma \vdash e : t$ where $e \equiv x$. By inversion, $x\bind t \in \Gamma$, so we can write $\Gamma \equiv \Gamma', x\bind t,\Gamma''$ and by repeated inversion of {\sc WFE-Bind}, $\vdash_w \Gamma',x\bind t$ and inverting again we get $\Gamma' \vdash_w t$. Inductively applying Lemma \ref{weakenings} gives us $\Gamma \vdash_w t$.

{\bf Case} {\sc T-Abs}: We have $\Gamma \vdash e : t$ where $e \equiv \lambda x.\, e'$ and $t \equiv \functype{x}{t_x}{t'}$. By inversion, we have $\Gamma, x\bind t_x \vdash e : t'$ and $\Gamma \vdash_w t_x$. By the inductive hypothesis, we have $\Gamma, x\bind t_x \vdash_W t'$. By rule {\sc WF-Func} we have $\Gamma \vdash_w \functype{x}{t_x}{t'}$.

{\bf Case} {\sc T-App}: We have $\Gamma \vdash e : t$ where $e \equiv e_1\; e_2$ and $t \equiv \existype{x}{t_x}{t'}.$ By inversion, $\Gamma \vdash e_1 : \functype{x}{t_x}{t'}$ and $\Gamma \vdash e_2 : t_x$. By the inductive hypothesis we have $\Gamma \vdash_w \functype{x}{t_x}{t'}$ and $\Gamma \vdash_w t_x$. By inverting rule {\sc WF-Func} (on the former judgement), we have $\Gamma \vdash_w t_x$ and $\Gamma,x\bind t_x \vdash_w t'$. By rule {\sc WF-Exis}, $\Gamma \vdash_w \existype{x}{t_x}{t'}$.

{\bf Case} {\sc T-Let}: We have $\Gamma \vdash e : t$ where $e \equiv \letin{x}{e_x}{e'}$. By inversion we have, in particular, that $\Gamma \vdash_w t$.

{\bf Case} {\sc T-Ann}: We have $\Gamma \vdash e : t$ where $e \equiv e'\col t$. By inversion we have $\Gamma \vdash e' : t$ and by the inductive hypothesis we conclude $\Gamma \vdash_w t$.

{\bf Case} {\sc T-Sub}: We have $\Gamma \vdash e : t$. By inversion we have, in particular, $\Gamma \vdash_w t$.
\end{proof}


\begin{lemma}\label{witness-sub}
(Witnesses and subtyping) If $\Gamma \vdash e_x : t_x$ and $\Gamma, x\bind t_x \vdash_w t'$ then $\Gamma \vdash t'[e_x/x] <: \existype{x}{t_x}{t'}$.
\end{lemma}
\begin{proof}
By Lemma \ref{sub-refl}, we have that $\Gamma \vdash t' <: t'$ and by the Substitution Lemma we have $\Gamma \vdash t'[e_x/x] <: t'[e_x/x]$.  Applying rule {\sc S-Witn} (with $t\equiv \existype{x}{t_x}{t'}$), we get $\Gamma \vdash t'[e_x/x] <: \existype{x}{t_x}{t'}$.
\end{proof}


\begin{theorem}\label{progress}
(The Progress Theorem) If $\varnothing \vdash e : t$ then either $e$ is a value or there exists a term $e'$ such that $e \hookrightarrow$ e'.
\end{theorem} {\bf NB: This theorem is bottomless.}

\begin{proof} We proceed by induction on the derivation tree of the judgment $\varnothing \vdash e : t$.

{\bf Case} {\sc T-Prim}: This case holds trivially because $e \equiv c$ is a value.

{\bf Case} {\sc T-Var}: This case cannot occur because $\Gamma = \varnothing$.

{\bf Case} {\sc T-Abs}: This case holds trivially because $e \equiv \lambda x.e'$ is a value.

{\bf Case} {\sc T-App}: We have $\varnothing \vdash e : t$ where $e \equiv e_1\; e_2$ and $t \equiv \existype{x}{t_x}{t'}$. By inversion, $\varnothing \vdash e_1 : \functype{x}{t_x}{t'}$ and $\varnothing \vdash e_2 : t_x$ for some type $s$. We split on five cases for the structure of $e_1$ and $e_2$.

First, consider $e_1 \equiv c$ and $e_2 \equiv v$; then by rule {\sc E-Prim} $e \equiv c\; v \hookrightarrow \delta(c,v)$, which is defined by Lemma \ref{prim-typing}.
Second, consider $e_1 \equiv c$ and $e_2$ not a value. By the inductive hypothesis (applied to $\varnothing \vdash e_2 : t_x$), there exists a term $e'_2$ such that $e_2 \step e'_2$. Thus $e_1\; e_2 \step e_1 \; e'_2$ by rule {\sc E-App1}.

Third, consider $e_1 \equiv \lambda x.e'_1$ and $e_2 \equiv v$. Then by the operational semantics, $\lambda x.e_2 \; v \step e_2[v/x]$. Fourth, consider $e_1 \equiv \lambda x.e'_1$ and $e_2$ not a value. By the inductive hypothesis, there exists a term $e'_2$ such that $e_2 \step e'_2$. Thus $e_1\; e_2 \step e_1 \; e'_2$ by {\sc E-App1} again.

This exhausts all possible cases in which $e_1$ could be a value in the empty environment. So, finally, consider $e_1$ not a value. Then by the inductive hypothesis there exists $e'_1$ such that $e_1 \hookrightarrow e'_1$. By the operational semantics, $e_1\; e_2 \hookrightarrow e'_1\; e_2$.

{\bf Case} {\sc T-Let}: We have $\varnothing \vdash e : t$ where
$e \equiv (\letin{x}{e_1}{e_2})$. By inversion, $\varnothing \vdash e_1:t_x$ and $x\bind t_x\vdash e_2: t$. First, suppose that $e_1 \equiv v$. Then by rule {\sc T-LetV}, $\letin{x}{v}{e_2} \step e_2[v/x]$. Second, suppose that $e_1$ is not a value. Then by the inductive hypothesis (applied to judgement $\varnothing \vdash e_1:t_x$), there exists a term $e'_1$ such that $e_1 \step e'_1$. Then by rule {\sc E-Let} we have $\letin{x}{e_1}{e_2} \step \letin{x}{e'_1}{e_2}$.

{\bf Case} {\sc T-Ann}: We have $\varnothing \vdash e : t$ where $e \equiv (e_1\col t)$. By inversion, $\varnothing \vdash e_1 : t$. By the inductive hypothesis either $e_1 \equiv v$ a value or there exists $e'_1$ such that $e_1 \hookrightarrow e'_1$. In the former case $(v\col t) \hookrightarrow v$ and in the latter case $(e_1\col t) \hookrightarrow (e'_1\col t)$.

{\bf Case} {\sc T-Sub}: We have $\varnothing \vdash e : t$. By inversion, $\varnothing \vdash e : s$, $\varnothing \vdash s <: t$, and $\varnothing \vdash_w t$ for some type $s$. By the inductive hypothesis, either $e$ is a value or there exists $e'$ such that $e \hookrightarrow e'$ and we are done.
\end{proof}

\begin{theorem}(The Preservation Theorem)
If $\varnothing \vdash e : t$ and $e \hookrightarrow e'$, then $\varnothing \vdash e' : t$.	
\end{theorem} {\bf NB: This theorem is bottomless. I need to prove the substitution lemma. I need to prove that the types in all judgements are well-formed. I need to prove that $<:$ is reflexive}
\begin{proof} We proceed by induction on the derivation tree of the judgment $\varnothing \vdash e : t$.

{\bf Case} {\sc T-Con}: Holds trivially because if $e \equiv c$ then there does not exist $e'$ such that $c \hookrightarrow e'$.

{\bf Case} {\sc T-Var}: Holds trivially because if $e \equiv x$ then there does not exist $e'$ such that $x \hookrightarrow e'$.

{\bf Case} {\sc T-Abs}: Holds trivially because if $e \equiv \lambda x.e_1$ then there does not exist any $e'$ such that $\lambda x.e_1 \hookrightarrow e'$.

{\bf Case} {\sc T-App}: We have $\varnothing \vdash e : t$ where $e \equiv e_1\; e_2$ and $t \equiv \existype{x}{t_x}{t'}$ for some variable $x$ and type $t_x$. By inversion, $\varnothing \vdash e_1 : \functype{x}{t_x}{t'}$ and $\varnothing \vdash e_2 : t_x$. We split on five cases for the structure of $e_1$ and $e_2$.

First, consider $e_1 \equiv c$ and $e_2 \equiv v$; then by the semantics $e' = \delta(c,v)$ and by Lemma \ref{prim-typing}, we have $\varnothing \vdash \delta(c,v) : t'[v/x]$ because n. By Lemma \ref{types-wf}, we have $\varnothing \vdash_w \existype{x}{t_x}{t'}$ and by inverting {\sc WF-Exis} we have $x\bind t_x \vdash_w t'$. By Lemma \ref{witness-sub}, $\varnothing \vdash t'[v/x] <: \existype{x}{t_x}{t'}$, and so by rule {\sc T-Sub}, $\varnothing \vdash \delta(c,v) : \existype{x}{t_x}{t'} $.

Second, consider $e_1 \equiv c$ and $e_2$ not a value. By Theorem \ref{progress}, there exists a term $e'_2$ such that $e_2 \step e'_2$. By rule {\sc E-App2}, $c\; e_2 \step c\; e'_2$ and by the determinism of the operational semantics, $e' \equiv c\; e'_2$.
By the inductive hypothesis, $\varnothing \vdash e'_2 : t_x$. We conclude by {\sc T-App} that $\varnothing \vdash e' : \existype{x}{t_x}{t'}$.

Third, consider $e_1 \equiv \lambda x.e'_1$ and $e_2 \equiv v$. Then $e_1\; e_2 \step e'_1[v/x]$ and by determinism of the operational semantics, $e' \equiv e'_1[v/x]$. By inversion of {\sc T-Abs}, we have $x\bind t_x \vdash e'_1 : t'$, and by the substitution lemma we have $\varnothing \vdash e'_1[v/x] : t'[v/x]$. By Lemma \ref{types-wf}, we have $\varnothing \vdash_w \existype{x}{t_x}{t'}$ and by inverting {\sc WF-Exis} we have $x\bind t_x \vdash_w t'$. By Lemma \ref{witness-sub}, $\varnothing \vdash t'[v/x] <: \existype{x}{t_x}{t'}$, and so by rule {\sc T-Sub}, $\varnothing \vdash e' : \existype{x}{t_x}{t'}$.

Fourth, consider $e_1 \equiv \lambda x.e'_1$ and $e_2$ not a value. By Theorem \ref{progress}, there exists a term $e'_2$ such that $e_2 \step e'_2$. By rule {\sc E-App2}, $(\lambda x.e'_1)\; e_2 \step (\lambda x.e'_1)\; e'_2$ and by the determinism of the operational semantics, $e' \equiv (\lambda x.e'_1)\; e'_2$.
By the inductive hypothesis, $\varnothing \vdash e'_2 : t_x$. We conclude by {\sc T-App} that $\varnothing \vdash e' : \existype{x}{t_x}{t'}$.

This exhausts all possible cases in which $e_1$ could be a value in the empty environment. So, finally, consider $e_1$ not a value.
Then by Theorem \ref{progress}, there exists an $e'_1$ such that $e_1 \hookrightarrow e'_1$. By determinism of the operational semantics, $e' \equiv e'_1\; e_2$. By the inductive hypothesis,
$\varnothing \vdash e'_1 : \existype{x}{t_x}{t'}$. By rule {\sc Syn-App}, $\varnothing \vdash e' : \existype{x}{t_x}{t'}$.

{\bf Case} {\sc T-Let}: We have $\varnothing \vdash e : t$ where $e \equiv (\letin{x}{e_1}{e_2})$ and $t \equiv t_2$. By inversion,
$\varnothing \vdash e_1 : t_1$,\quad $x\bind t_1\vdash e_2 : t_2$, and $\varnothing \vdash_w t_2$ for some type $t_1$. 
First suppose that $e_1$ is not a value. Then by Theorem \ref{progress}, there exists some term $e'_1$ such that $e_1 \step e'_1$. By Rule {\sc E-Let}, $\letin{x}{e_1}{e_2} \step \letin{x}{e'_1}{e_2}$, and by determinism of the operational semantics, $e' \equiv \letin{x}{e'_1}{e_2}$. By the inductive hypothesis, $\varnothing \vdash e_1 : t_1$. Then by {\sc T-Let}, $\varnothing \vdash e' : t_2$.

Second, suppose that $e_1 \equiv v$, for some value $v$. Then by rule {\sc E-LetV}, $\letin{x}{v}{e_2}\step e_2[v/x]$. By determinism of the operational semantics, $e' \equiv e_2[v/x]$. By the substitution lemma, $\varnothing \vdash e_2[v/x] : t_2[v/x]$. But by $\varnothing vdash_w t_2$, we know that $x$ does not appear free in $t_2$ so $t_2[v/x] = t_2$ and $\varnothing \vdash e' : t_2$.

{\bf Case} {\sc T-Ann}: We have $\varnothing \vdash e : t$ where $e \equiv (e_1 : t)$ and $e \hookrightarrow e'$. By inversion,
$\varnothing \vdash e_1 : t$. By Theorem \ref{progress} there exists $e'_1$ such that $e_1 \hookrightarrow e'_1$. By rule {\sc E-Ann} $(e_1 : t) \hookrightarrow (e'_1 : t)$ and by the determinism of the operational semantics we must have $e' \equiv (e'_1 : t)$. Then by the inductive hypothesis, $\varnothing \vdash e'_1 : t$. By rule {\sc Syn-Ann}, $\varnothing \vdash (e'_1 : t) : t$. 

{\bf Case} {\sc T-Sub}: We have $\varnothing \vdash e : t$. By inversion $\varnothing \vdash e : s$ and $\varnothing \vdash s <: t$ for some type $s$, and also $\varnothing \vdash_w t$. By the inductive hypothesis $\varnothing \vdash e' : s$. By rule {\sc Chk-Syn}, $\varnothing \vdash e' : t$.
\end{proof}

\section{Algorithmic Typing} %%%%%%%% 444444 %%%%%%%%%%%%%%

\begin{lemma}\label{SMT-Entails}
(Denotational Soundness of SMT) If $\Gamma \vdash p$ then $\Gamma \vdash_e p$.
\end{lemma}


\newpage
General Todo list:
\begin{itemize}
	\item does having call-by-value (non-lazy) semantics affect wha the definitions of denotations of types ($\lb \; \rb$) ought to be? Should any expressions being substituted into types become values?
	\item In Lemma 5, case {\sc T-Abs} requires some additional lemma like: if $e \many v$ then $\lb t[v/x] \rb \subseteq \lb t[e/x] \rb$
	\item Lemma 5 needs a part (3): If $\Gamma \vdash_w t$ then $\foralltheta \Rightarrow \varnothing \vdash_w \theta(t)$.
	\item The entailment rules don't quite seem to fit with the denotation definitions. Can't prove case {\sc S-Base} in the Substitution Lemma (Lemma 6).
	
\end{itemize}
\end{document}